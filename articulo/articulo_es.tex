\\documentclass[12pt, a4paper]{article}\n\\usepackage[utf8]{inputenc}\n\\usepackage{geometry}\n\\usepackage{hyperref}\n\n\\geometry{a4paper, margin=1in}\n\n\\title{\n    {\\huge Clima Criminal: Una Herramienta Geoespacial para el Análisis de la Criminalidad en la República Dominicana} \\\\\n    \\vspace{0.5cm}\n    \\large Justificación y Propuesta Técnica\n}\n\n\\author{\n    Lic. Carlos G. Ramirez \\\\\n    \\texttt{carlos.ramirez@email.com} % Placeholder email\n    \\and\n    Gemini \\\\\n    \\textit{Modelo de Lenguaje Grande de Google}\n}\n\n\\date{\\today}\n\n\\begin{document}\n\n\\maketitle\n\n\\begin{abstract}\nLa República Dominicana, a pesar de los avances en la reducción de ciertas tasas delictivas, enfrenta un desafío persistente en cuanto a la criminalidad y la alta percepción de inseguridad ciudadana. Los datos estadísticos, aunque disponibles, se encuentran a menudo dispersos y presentados en formatos que dificultan su análisis por parte del público general, investigadores y organizaciones de la sociedad civil. Este artículo presenta \\textbf{Clima Criminal}, una aplicación web interactiva diseñada para centralizar, visualizar y analizar datos de criminalidad a nivel nacional. La plataforma ofrece herramientas geoespaciales como mapas de calor y coropléticos, filtros de datos dinámicos y la generación de reportes, con el objetivo de democratizar el acceso a la información, fomentar la transparencia y proveer un recurso valioso para el estudio de patrones delictivos en el país.\n\\end{abstract}\n\n\\section{Introducción: El Problema de la Inseguridad}\n\nLa seguridad ciudadana es un pilar fundamental para el desarrollo social y económico de cualquier nación. En la República Dominicana, la delincuencia es una de las principales preocupaciones de la población. Según datos oficiales, en 2023 el país registró una tasa de homicidios de \\textbf{11.5 por cada 100,000 habitantes}. Si bien esta cifra representa una disminución con respecto a años anteriores, el robo se mantiene como el delito de mayor incidencia, con más de 78,000 casos reportados en el mismo año.\n\nMás allá de las cifras oficiales, la \\textit{percepción de inseguridad} tiene un impacto tangible y profundo en la sociedad. Encuestas revelan que una gran parte de los dominicanos ha modificado sus rutinas diarias por temor a ser víctima de un delito, afectando la cohesión social y la confianza en las instituciones. Esta situación crea una demanda crítica de información clara, accesible y detallada sobre la criminalidad.\n\n\\section{Justificación de la Necesidad}\n\nLa necesidad de una herramienta como \\textbf{Clima Criminal} se fundamenta en tres brechas principales:\n\n\\begin{enumerate}\n    \item \\textbf{Fragmentación de Datos:} La información sobre criminalidad es publicada por diversas entidades gubernamentales (Policía Nacional, Procuraduría General, Oficina Nacional de Estadística). Esto obliga a los interesados a consultar múltiples fuentes, a menudo con formatos y metodologías distintas, dificultando la obtención de un panorama integrado.\n    \n    \item \\textbf{Falta de Análisis Geoespacial Accesible:} Los reportes estadísticos tradicionales suelen presentar datos agregados a nivel nacional o provincial. Carecen de la granularidad de un mapa interactivo que permita identificar \\textit{dónde} se concentran los incidentes, un factor clave para que los ciudadanos y las autoridades locales puedan tomar decisiones informadas sobre riesgos específicos.\n    \n    \item \\textbf{Distancia entre Datos y Percepción Pública:} Una plataforma de datos abierta y fácil de usar puede ayudar a cerrar la brecha entre las estadísticas oficiales y la percepción pública. Al permitir que cualquier persona explore los datos por sí misma, se fomenta un debate público más informado y se fortalece la transparencia.\n\\end{enumerate}\n\n\\section{La Solución Propuesta: Clima Criminal}\n\nPara abordar estas necesidades, se ha desarrollado la aplicación \\textbf{Clima Criminal}. Se trata de una plataforma web construida sobre un stack tecnológico moderno (Django, JavaScript, Leaflet.js) que ofrece las siguientes funcionalidades clave:\n\n\\begin{itemize}\n    \item \\textbf{Visualización Interactiva:} Mapas de calor para ver la concentración de crímenes y mapas coropléticos que ilustran la distribución de la criminalidad por provincia.\n    \item \\textbf{Filtros Dinámicos:} Permite a los usuarios filtrar los datos por tipo de crimen, provincia y rangos de fecha, ofreciendo una personalización completa del análisis.\n    \item \\textbf{Reporte Ciudadano:} Un formulario para que los ciudadanos puedan reportar incidentes, contribuyendo a una base de datos más rica (actualmente como prueba de concepto).\n    \item \\textbf{Generación de Estadísticas y Reportes:} Gráficos dinámicos y la capacidad de generar reportes imprimibles de los datos filtrados, tanto a nivel general como provincial.\n\\end{itemize}\n\nLa arquitectura de la aplicación está diseñada para ser escalable, permitiendo en el futuro la integración de nuevas fuentes de datos y módulos de análisis más avanzados.\n\n\\section{Conclusión}\n\n\\textbf{Clima Criminal} no es solo un visualizador de datos, sino una iniciativa para empoderar a la sociedad dominicana con información accesible y contextualizada. Al proporcionar herramientas para el análisis geoespacial de la criminalidad, la plataforma sirve como un recurso valioso para ciudadanos, académicos, periodistas y responsables de políticas públicas, promoviendo una mayor comprensión de la dinámica de la seguridad en la República Dominicana y fomentando una cultura de toma de decisiones basada en evidencia.\n\n\\begin{thebibliography}{9}\n    \\bibitem{ONE2023}\n    Oficina Nacional de Estadística (ONE), República Dominicana. (2024). \\textit{Anuarios Estadísticos}.\n    \n    \\bibitem{PoliciaNacional2024}\n    Policía Nacional de la República Dominicana. (2024). \\textit{Informes y Estadísticas Delictivas}.\n    \n    \\bibitem{Percepcion}\n    Estudios de percepción de inseguridad ciudadana (varias fuentes periodísticas y académicas, 2023-2024).\n\n\\end{thebibliography}\n\n\\end{document}\n